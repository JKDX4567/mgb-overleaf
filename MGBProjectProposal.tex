\chapter{Project Proposal\\ 
\small{\textit{-- Matthew Smith, Bowen Jiang, Gleb Myshkin}}}
\label{Chapter!Project Proposal}
\index{Project Proposal}
\index{Chapter!Project Proposal}


    Please describe your project.
    Include a title and a description that includes sample task and DevSecOps tools used, such as source control, testing, deployment, databases, etc.  At this point everything is very vague, but I want you to think about the tools you might need, even if you don't have all the tools very detailed yet.

\section{Project description}

\hspace {2.5em} Our project aims to allow students and programmers to easily create uml style diagrams, and export them as an eps file ready to be used however the user sees fit. Titled MGB-UML, this will be done by creating an app that allows the user to drag on specific "blocks", connect, and edit them to display whatever the developer needs them to display for their project. The main focus of this project is to tackle the lackluster options that handle uml diagram creation,  creating an easier to use system that handles dynamic needs, ensuring that users have a comfortable time creating uml diagrams, being adaptable for those in the know, and allowing those who aren't knowledgeable to have a guide to help them practice this vital skill.

\section{DevSecOps Tools}
\hspace {2.5em} To build and maintain the application, we plan to use a set of DevSecOps tools that cover source control, testing, deployment, and security. First of all, we will manage the source control using GitHub, and will use Docker to make our deployment easier across different operating systems. For diagram generation and export, we will rely on TikZ to produce high-quality EPS graphics and clickable reference in overleaf and Umlet as an open-source reference for drag-and-drop UML editing. We'll also look into TikZiT as an open-source tool that directly works with tikZ and a pdflatex compiler such as TinyTex. We might also look into using GitHub to sync our local and Overleaf file progress. A Kanban board in Jira will guide our workflow, helping us manage tasks, track progress. Together, these tools will streamline coding, testing, and deployment while ensuring that our UML-to-EPS project process remains secure.
